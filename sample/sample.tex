% このtexファイルはサンプルです。
\documentclass[uplatex]{jsarticle}
\usepackage{amsmath}
% 参考文献: upBibTeX + jplain.bst
\usepackage{cite}
% クリック可能リンク (uplatex + dvipdfmx)
\usepackage[dvipdfmx,unicode,bookmarks=true]{hyperref}
\usepackage{pxjahyper}
\hypersetup{
  colorlinks=true,
  linkcolor=blue,
  urlcolor=blue,
  citecolor=blue,
  bookmarksnumbered=true
}

\title{upLaTeX + upBibTeX + jplain.bst テンプレート}
\author{山田太郎}
\date{\today}

\begin{document}

\maketitle

\section*{概要}
この文書は\cite{fujita2020} に基づき、upBibTeX + jplain による文献管理を説明する。

\section*{結論}
Biber を使えば文献の柔軟な出力が可能である。

\bibliographystyle{jplain}
\bibliography{refs}

\end{document}
